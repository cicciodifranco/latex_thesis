\chapter{Rquisiti}
Dovrà essere progettato un insieme di componenti software per la gestione real-time dell'occupazione di diversi stalli di parcheggio identificati in deverse aree geografiche.
Una serie di stalli di parcheggio sono monitorati da una videocamera connessa a internet e, tramite l'elaborazione dell'immagine, in tempo reale sarà notificato se uno stallo verrà liberato o occupato. Per facilitare la gestione  delle coordinate geografiche verrà usata la tecnica di clusteriing geografica denominata ``geohash" che suddivide in una griglia l'intero emisfero e permette di identificare un'area geografica con un'arbitraria precisione tramite una sequenza di bit. Sarà possibile reperire, dopo necessaria autenticazione, un insieme di stalli di parcheggio indicando il geohash interessato. Le informazioni che dovranno essere memorizzate per stalli e videocamere sono, oltre ad un id univoco e mnemonico, coordinate GPS e geohash ricavato dalle coordinate. In ogni entità oltre alle informazioni sulla localizzazione saranno presenti le rispettive relazioni relative al monitoraggio: ogni stallo di parcheggio dovrà contenere informazioni sulle videocamere che lo stanno monitorando e ogni videocamera dovrà contenere informazioni sugli stalli di parcheggio che sta monitorando.
Tutte queste informazioni dovranno essere memorizzate in maniera persistente in una base dati e a supporto di questa dovrà essere presente un meccanismo di cache per velocizzare le comunicazioni.

Ogni utente dopo aver effettuato il login e aver ottenuto un token di sicurezza potrà reperire le informazioni relative a gli stalli di parcheggio contenti nel geohash corrispondente alla posizione occupata dall'utente. Ad ogni richiesta il client fornirà il token che ha ottenuto tramite il login e la propria posizione GPS dalla quale verrà calcolato il geohash.

Dopo che la procedura di login sarà andata a buon fine e dopo aver reperito le informazioni relative alla propria posizione il client può creare un collegamento con il server tramite web-socket per ricevere gli aggiornamenti sugli stalli. Sarà previsto pure un handshake token based tramite web-socket.

Una videocamera quando rileva il cambio di stato di uno stallo di parcheggio, dopo previa autenticazione, dovrà notificare l'evento fornendo l'id dello stallo e il nuovo stato (``free" o ``busy").



\section{Scalabilità delle risorse}

\section{Ottimizzazione dell' I/O sul DB}

\subsection{Politiche di gestione della cache}

\section{Notifica utenti} 