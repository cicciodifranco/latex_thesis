\chapter{Conclusione e sviluppi futuri}


Con lo sviluppo di questo progetto è stato creato un prototipo applicativo completo per la gestione real-time dell'occupazione degli stalli di parcheggio monitorati. 
\newline

Tale applicativo è tuttora in fase di evoluzione poiché si stanno valutando nuove tecnologie proposte da AWS in merito alla gestione della comunicazione tra i sistemi embedded e l'architettura Cloud Park Smart. 
\newline

L'analisi dei requisiti è servita a tracciare le linee guida dello sviluppo e a tal proposito si è applicata una metodologia elastica di sviluppo del software, incentrata sul raggiungimento delle ``feature" di prodotto, con una ciclicità nella ridefinizione dell'architettura complessiva del sistema.

A tal proposito, i principi fondamentali della metodologia Agile sono stati di vitale importanza nella gestione complessiva del progetto.
In tal modo si è stati pronti a rispondere ai cambiamenti oltre che all'aderire alla pianificazione, modificando le priorità di lavoro nel rispetto dell'obiettivo finale \footnote{\href{http://www.agilemanifesto.org}{Agile Manifesto}}.
Attualmente il software prodotto è impiegato attivamente all'interno del progetto Park Smart.	


Allo stato attuale il progetto Park Smart è ancora in fase di sviluppo e si stanno valutando nuove tecnologie per ampliare le ``feature" di prodotto e migliorare la \textit{user experience}. 

Attualmente si sta sviluppando il modulo per la comunicazione delle videocamere tramite MQTT autenticate con il Cloud ParkSmart con un certificato digitale X.509.
\newline

Verrà prodotta una dashboard web, con molteplici funzionalità collegate al cloud ParkSmart, dalla quale si potranno gestire tutte le informazioni su:
\begin{itemize}
	\item Telecamere;
	\item Stalli di parcheggio;
	\item Partner aziendali;
	\item Partner tecnici;
	\item Impianti installati;
\end{itemize}


Inoltre saranno previste funzionalità di monitoraggio per verificare lo stato degli impianti installati.

