\chapter{Conclusione}


Con lo sviluppo di questo progetto è stato creato un prototipo applicativo completo per la gestione real-time dell'occupazione degli stalli di parcheggio monitorati. Tale applicativo è tuttora in fase di evoluzione poichè si stanno valutando nuove tecnologie proposte da AWS in merito alla gestione della comunicazione tra i sistemi embedded e l'architettura cloud Park Smart. L'analisi dei requisiti è servita a tracciare le linee guida dello sviluppo e a tal proposito si è applicata una metodologia elastica di sviluppo del software, incentrata sul raggiungimento delle ``feature" di prodotto, con una ciclicità nella ridefinizione dell'architettura complessiva del sistema.

A tal proposito, i principi fondamentali della metodologia Agile sono stati di vitale importanza nella gestione complessiva del progetto.
In tal modo si è stati pronti a rispondere ai cambiamneti oltre che all'aderire alla pianificazione, modificando le priorità di lavoro nel rispetto dell'obiettivo finale \footnote{\href{http://www.agilemanifesto.org}{Agile Manifesto}}. 

Attualmente il software prodotto è impiegato attivamente all'interno del progetto Park Smart 	