\chapter{Introduzione}
L'elaborato si propone di presentare un progetto aziendale, promosso da Park Smart srl\footnote{\href{http://www.parksmart.it}{www.parksmart.it}}, relativo alla gestione real-time di eventi geolocalizzati. 

Park Smart è una startup che opera nel campo delle Smart Cities, proponendo un progetto innovativo il cui scopo è monitorare gli stalli di parcheggio bordo strada e comunicare ai cittadini, mediante un'apposita applicazione per smartphone, dove trovare parcheggio, migliorando la qualità della vita in città.

La soluzione proposta offre un sistema composto da tre componenti fondamentali:

\begin{enumerate}
\item sistema di video analisi;
\item una piattaforma orientata ai Web-Services per 
la gestione degli eventi;
\item una applicazione mobile per l'utente finale.
\end{enumerate}


In questo elaborato verrà trattato il sistema di gestione real-time degli eventi: quando una telecamera, preposta al monitoraggio di una serie di stalli di parcheggio, rileva un cambio di stato, si occuper\`a di notificarlo al sistema Cloud Park Smart.

Tale evento sarà elaborato e notificato all'utente finale in tempo reale quando si trova in prossimit\`a della zona dove si è verificato l'evento. 

Le scelte tecnologiche fatte sono frutto di un accurata analisi comparativa e tutti i moduli software descritti in questo documento sono stati progettati e sviluppati dall'autore. Sotto supervisione dell'azienda ogni passo dello sviluppo ha seguito un iter di validazione composto da due fasi di test.

%quali sono le fasi di test?


