\documentclass[a4paper, 11pt]{report}
%\usepackage[latin1]{inputenc}
\usepackage{graphicx}
\usepackage{hyperref}
\usepackage{booktabs}
\usepackage{enumitem}
\usepackage[italian]{babel}
\usepackage[utf8x]{inputenc}
\usepackage{algpseudocode}
\usepackage{algorithm}
\usepackage{algorithmicx}
\usepackage{fullpage}
\usepackage{listings}
\usepackage{color}
\definecolor{lightgray}{rgb}{.9,.9,.9}
\definecolor{darkgray}{rgb}{.4,.4,.4}
\definecolor{purple}{rgb}{0.65, 0.12, 0.82}
\lstdefinelanguage{JavaScript}{
  keywords={typeof, new, true, false, catch, function, return, null, catch, switch, var, if, in, while, do, else, case, break, this},
  keywordstyle=\color{blue}\bfseries,
  ndkeywords={class, export, boolean, throw, implements, import},
  ndkeywordstyle=\color{darkgray}\bfseries,
  identifierstyle=\color{black},
  sensitive=false,
  comment=[l]{//},
  morecomment=[s]{/*}{*/},
  commentstyle=\color{purple}\ttfamily,
  stringstyle=\color{red}\ttfamily,
  morestring=[b]',
  morestring=[b]"
}

\lstset{
   language=JavaScript,
   backgroundcolor=\color{lightgray},
   extendedchars=true,
   basicstyle=\footnotesize\ttfamily,
   showstringspaces=false,
   showspaces=false,
   numbers=none,
   numberstyle=\footnotesize,
   numbersep=9pt,
   tabsize=2,
   breaklines=true,
   showtabs=false,
   captionpos=b
}

\lstdefinelanguage{HTTP}{
  keywords={GET, POST, PUT, DELETE, OPTIONS, Number, String, Boolean, HTTP},
  keywordstyle=\color{blue}\bfseries,
  ndkeywords={Params},
  ndkeywordstyle=\color{purple}\bfseries,
  identifierstyle=\color{black},
  sensitive=false,
  comment=[l]{//},
  morecomment=[s]{/*}{*/},
  commentstyle=\color{purple}\ttfamily,
  stringstyle=\color{red}\ttfamily,
  morestring=[b]',
  morestring=[b]"
}


\title{Sistema cloud based real-time \\per la gestione di eventi geolocalizzati.}
\author{Di Franco Francesco}
\date{09-03-2016}


\begin{document}
\begin{titlepage}
\begin{center}


\noindent\begin{minipage}{0.2\textwidth}% adapt widths of minipages to your needs
\includegraphics[scale=0.25, bb=0 0 200 200]{./img/logo_unict2.png}
\vspace{0.4truecm}
\end{minipage}%
%\hfill%
\begin{minipage}{0.8\textwidth}
\begin{center}
\Large{	Università degli Studi di Catania \\
			Dipartimento di Matematica e Informatica \\
			Corso di Laurea in Informatica triennale }

\vspace{0.4truecm}
\end{center}
\end{minipage}
\hbox to \textwidth{\hrulefill}

\vspace{1.5truecm}
\Large {\sc Francesco Di Franco}
\vspace{1.0truecm}


\LARGE{\sc {\textbf{Sistema cloud based per la gestione real-time di eventi geolocalizzati.}}}


\vspace{2.0truecm}
\begin{minipage}[t][3cm][b]{0,5\textwidth}
\hbox to \textwidth{\hrulefill}
\begin{center}
\vspace{0,6truecm}
Relazione progetto finale
\end{center}

\hbox to \textwidth{\hrulefill}
\end{minipage}


\vspace{3.0truecm}
\large
\begin{flushright}
Relatore \\
{\sc {\textbf{Chiar.mo Prof. S. Riccobene }}}
\\
\vspace{1truecm}
Correlatore \\
{\sc {\textbf{Dott. G. Patanè}}}
\end{flushright}

\vspace{0,8truecm}

\hbox to \textwidth{\hrulefill}
\Large
Anno Accademico 2015/16
\end{center}
\end{titlepage}

\thispagestyle{empty}
\cleardoublepage






\chapter*{Indice}

\addcontentsline{toc}{chapter}{Indice}
\chaptermark{Indice}

\large
\begin{enumerate}[label*=\arabic*.]
\item Introduzione %1

\item Obiettivi %2
\begin{enumerate}[label*=\arabic*.]
	\item Problematiche affrontate %2.1
	
\end{enumerate}

\item Requisiti %3
\begin{enumerate}[label*=\arabic*.]
	\item Scalabilit\`a delle risorse %3.1
	\item Ottimizzazione dell' I/O sul DB e gestione della cache %3.2
	\item Notifica utenti %3.3
\end{enumerate}

\item Tecnologie %4
\begin{enumerate}[label*=\arabic*.]
	\item Analisi comparativa %4.1
	\begin{enumerate}[label*=\arabic*.]
		\item Basi di dati %4.1.1
		\item Web services e Server Side Framework %4.1.2
	\end{enumerate}

	\item Tecnologie selezionate %4.2
	\begin{enumerate}[label*=\arabic*.]
		\item AWS: Elastic Beanstalk %4.2.1
		\item AWS: Redis  (All in-memory key-value DB)%4.2.2
	\end{enumerate}
\end{enumerate}

\item Progettazione del software %5
\begin{enumerate}[label*=\arabic*.]
	\item Classificazione eventi %5.1i
	\item Database (AWS: DynamoDB)
	\item NodeJs 
	\item Redis
	\item Socket.io (Web socket)
\end{enumerate}

%\item Architettura
%\begin{enumerate}[label*=\arabic*.]
%	\item Architettura 1 (NodeJs, Redis, Socket.io)
%	\item Architettura 2 (AWS API Gateway, AWS IoT, AWS Lambda)
%	\item Architetture a confronto
%\end{enumerate}

\item Workflow, testing e tecnologie di supporto
\item Conclusione
\item Sviluppi futuri
\end{enumerate}
%chapter
\chapter{Introduzione}
L'elaborato si propone di presentare un progetto aziendale, promosso da Park Smart srl \footnote{\href{http://www.parksmart.it}{www.parksmart.it}}, relativo alla gestione real-time di eventi geolocalizzati. Park Smart è una startup che opera nel campo delle smart cities, proponedo un progetto innovativo il cui scopo è semplificare la vita in citt\`a, aiutando chi \`e alla ricerca di un parcheggio per la propria auto. 

La soluzione proposta offre un sistema composto da tre componenti fondamentali:

\begin{enumerate}
\item Sistema di videoanalisi.
\item Una piattaforma web-services oriented per 
la gestione degli eventi.
\item Applicazione mobile per l'utente finale.
\end{enumerate}

%\newline
In questo elaborato verrà trattato il sistema di gestione real-time degli eventi: qunado una telecamera, addetta al monitoraggio di una serie stalli di parcheggio, tramite avanzati algoritmi di videoanalisi, rileva un cambio di stato si occuper\`a di notificarlo. Tale evento sarà elaborato e notificato all'utente finale in prossimit\`a dell'evento stesso. 

Le scelte tecnologiche fatte sono frutto di un accurata analisi comparativa e tutti i moduli software descritti in questo documento sono stati progettati e sviluppati dall'autore. Sotto supervisione dell'azienda ogni passo dello sviluppo ha seguito un iter di validazione composto da due fasi di test.

\chapter{Obiettivi}
In questo elaborato verranno descritte le principali tecnologie usate e le valutazioni tecniche che hanno portato alle scelta delle stesse. Alcune scelte come il sevice provider o la specifica tecnologia di Database Managment System (DBMS) sono state influenzate da fattori interni all'azienda motivati non solo da fattori tecnologici ma anche da particolari partnership e/o aggevolazioni di natura economica.


\section{Problematiche affrontate}
La crezione di un sistema real-time basato su tecnologie cloud per la gestione di eventi geolocalozzati necessita di diverse valutazione e in particolare sono stati attenzionati i seguenti punti:
\begin{itemize}
\item Scalabilità delle risorse
\item Latenza tra il verificarsi dell'evento e la notifica degli utenti
\item Sicurezza
\end{itemize}

\section*{Scalabilità delle risorse}
L'avanzare delle tecnologie in ambito informatico e delle telecomunicazioni ha apportato diversi cambiamenti nell'ambiente dei sistemi distribuiti e in particolare alla nascita di quello che oggi chiamiamo Cloud.
Per cloud oggi si intende la modalità di strutturazione, orchestrazione ed erogazione on-demand di risorse informatiche tramite internet come storage, computing o hardware configurabile. Tali servizi sono completamente ospitati e configurati da provider che su richiesta, tramite procedure automatizzate, forniscono tali risorse in maniera rapida all'utente che le richiede.
I servizi cloud sono divisi in tre categorie fondamentali:
\begin{itemize}
\item SaaS (Software as a Service) - Servizi che consentono l'utilizzo di software installati su macchine remote
\item DaaS (Data as a Service) - Servizi di storage che consentono la memorizzazione persistente di dati su macchine remote
\item HaaS (Hardware as a Service) - Servizi che permettono l'elaborazione di dati in modo distribuito
\end{itemize}
I vantaggi che derivano dall'utilizzo di tali servizi sono la semplicità e la trasparenza con cui le risorse vengono erogate e/o aggiunte a sistemi già esistenti. Oggi la capacità di un sistema di adeguarsi al carico viene definità scalabilità cioè aumentare o diminuire le risorse in funzione delle necessità.Tale proprietà viene considerata un parametro di qualità determinante in numerosi ecosistemi hardware-software. La distribuzione del carico e la massimizzazione delle prestazioni è un fattore determinante per sistemi software destinati all'utilizzo su vasta scala.

L'architettura proposta fa uso per lo più di servizi SaaS altamente scalabili e il provider scelto è Amazon le cui tecnologie, denominate Amazon Web Services, sono tra le più utilizzate in ambito enterprise.


\section*{Latenza tra il verificarsi dell'evento e la notifca degli utenti}
Uno degli aspetti fondamentali per migliorare l'esperienza dell'utente è il fattore real-time cioè riuscir a visualizzare in tempo reale lo stato dello stallo di parcheggio interessato e nel contempo rimanere informato se lo stesso cambia di stato. La problematica più rilavante mentre si sta percorrendo un tragitto verso lo stallo interessato è appunto occupazione dello stallo da parte di un'altro automobilista. La notifica del cambio di stato in real-time è un aspetto molto curato che permette alla logica interna dell'applicazione per smartphone di cercare lo stallo libero più vicino a quello interessato. Per contattare l'utente  in caso di un eventuale aggiornamento verrà usata la tecnologia ``web-socket'' che offre un canale di comunicazione full-duplex tra client e server rendendo cosi molto più semplici le comunicazioni dal server verso il client. 
In tal modo si riesce a diminuire l' I/O riducendo le richieste (API calls) da parte del client e si fa in modo che il client sia contattato solo quando si verifichi un evento. 

\section*{Sicurezza}
Un aspetto che non può essere trascurato è quello della sicurezza e nello specifico le modalità di autenticazione e comunicazione.


Prima di poter cominciare una qualsiasi comunicazione ogni entità dovra essere autenticata e nello specifico le entità che dovranno comunicare e di conseguenza essere autenticate sono due:

\begin{itemize}
	\item Utenti
	\item Telecamere
\end{itemize}

\subsection*{Autenticazione} 
Formalmente per autenticazione si intende quel processo, durante una comunicazione, tramite il quale un entità verifica che l'altra parte sia realmente chi dichiara di essere.
Le due modalità di autenticazione sono profondamente diverse a causa della diversa natura delle entità. 

Un utente per usufruire del servizio dovrà registrarsi e di conseguneza cedere alcune sue informazioni personali per essere autenticato. In ausilio a questa procedura verrà utilizzato un sevizio AWS denominato ``Cognito'' che permette, dopo aver effettuato il login su un qualsiasi identity provider (Facebook, Google, Parksmart stessa), di verificare l'dentità dell'utente e mantenere la consistenza di un dataset personale utile alla sincronizzazione di più dispositivi. Cognito usa un maccanismo che verifica il token fornito dal dispositivo utente con l'identity provider e, dopo un eventuale conferma, inizializza l'identità e la registra in una pool alla quale fanno riferimento delle Access Control List (ACL) che specificano le capabilities della pool. In questo modo si ha la possibilità di gestire diversi gruppi di utenti assegnando o rimuovendo dinamicamente risorse specifiche.

L'autenticazione delle telecamere invece usa un meccanismo basato sulla certificazione digitale ``X.509'' che permette segrettezza e autenticazione, tali certificati sono denominati ``self signed'' poichè rilasciati dall'azienda stessa ed hanno validità solo all'interno dell'ecosistema Parksmart. In tal modo ogni entità che vuole inviare un aggiornamento dovrà essere fornita di un certificato valido per completare l'handshake iniziale. Grazie a questo meccanismo si garantisce segretezza nella comunicazione grazie alla crittografia asimmetrica data dallo standard X.509 e si riesce ad autenticare l'entità tramite la verifica del certificato.

\subsection*{Comunicazione}
Dopo che il processo di autenticazione è andato a buon fine l'utente potrà richiedere i dati relativi alla zona in prossimità della sua posizione. Il protocollo usato per questo tipo di cominicazioni sarà l' ``HTTPS'' (HTTP + TLS) che garantisce segretezza e integrità.

Per la comunicazione con le telecamere varrà usato un protocollo molto snello in termini di banda, denominato ``MQTT'' usa un meccanismo di publish/subscribe e trova molte applicazioni nell ``Internet of Things" (IoT).

\section{Obiettivi raggiunti}
Con lo sviluppo di questo progetto sono stati creati due prototipi applicativi completi per la gestione real-time dell'occupazione di dive



\chapter{Rquisiti}


\section{Classificazione eventi}


\section{Scalabilità delle risorse}

\section{Ottimizzazione dell' I/O sul DB}

\subsection{Politiche di gestione della cache}

\section{Notifica utenti} 
\chapter{Tecnologie}

Per portare a termine il progetto sono state valutate diverse tecnologie, e in particolare uno degli obbiettivi che ha influenzato l'analisi delle stesse è satto evitare o ridurre al minimo il ``provisoning",  cioè quel processo in cui un amministartore di sitema configura l'ambiente di lavoro/sviluppo e assegna risorse e permessi a gli utenti. Con le tencologie proposte da AWS questa particolare fase di configuarezione iniziale è molto ridotta e a volte assente. Per questo motivo circa il 90\% delle tecnologie sono ospitate da AWS di cui circa la metà completamente propietaria AWS.


\section{Analisi comparativa}
Le tecnlogie che sono state analizzate abbracciano tre categorie di software:
\begin{enumerate}
	\item Basi di dati
	\item Server side framework
	\item Web socket
\end{enumerate}

\subsection{Basi di dati}
Tra le tecnologie moderne possiamo trovare diverse soluzioni per la memorizzazione persistente di informazoini in un base dati e molte di queste, rispetto al classico approccio relazionale in cui ogni entità deve avere uno schema ben definito, offrono diversi vantaggi come:
\begin{itemize}
	\item Struttura ``schema less"
	\item Possibilità di esecuzione su un ambiente distribuito
	\item Possibilità di referenziare/includere altre entità
\end{itemize}
Questa categoria di database viene denominata ``Not only SQL" (NoSql), e tali sistemi si distunguono appunto dal modello classico per la loro flessibilità, scalabilità e alta resa in termini di presatzioni; requisiti che oggi sono reputati fondamentali per applicazioni enterprise.
Le tecnologie di DBMS prese in esame per questo progetto sono due e in particolare verranno analizzati DynamoDb di AWS e MongoDb di MongoDb Inc. 
\subsubsection*{DynamoDb}
DynamoDb è una soluzione enterprise completamente sviluppata e ospitata da AWS e questo rappresenta uno dei suoi principali vantaggi, infatti offre la possibilità di gestire le prestazioni e le risorse allocate in modo semplice da un pannello di controllo dedicato. DyanamoDb è schema less il che vuol dire che non esistono limiti sul numero di attributi che un record può avere e indipendentemente dagli altri ognuono può avare un numero arbitrario di attributi ad eccezzione di una chiave che deve essere univoca e presente per ogni record. DynamoDb supporta due tipologie di chiavi
\begin{itemize}
	\item Partition key (Hash key): tale chiave ha la funzione di una chiave primaria e prende il nome di hash key poichè serve de input ad una funzione hash (interna ad AWS) che calcola la posizione (partizione) dove il record è memorizzato.

	\item Partition key and sort key (Hash and range key): è una chiave primaria composta da due attributi, la prima è la partition key e la seconda, sort key, viene usata come chiave di ordinamento. Tutti i record con la stessa partition key saranno ordinati secondo la sort key. 
\end{itemize}
 Per facilitare le ricerche DynamoDb offre la possibilità di definire per ogni tabella ulteriori indici e in particolare:

\begin{itemize}
	\item 5 indici globali secondari (global secondary index)
	\item 5 indici locali secondari (local secondary index)
\end{itemize}
DynamoDb è un DBMS pensato per essere performante e altamente scalabile perciò non offre alcun tipo di referenziamento con altre tabelle e di conseguenza tutte le operazione ad esse correllate come join o transazioni.
Le operazioni che si possono effettuare con DynamoDb sono le seguenti:
\begin{itemize}
	\item GetItem : data la primary key (Hash) viene reperito il record corrispondente.
	\item PutItem : inserisce un nuovo record nel database.
	\item Query : reperisce record secondo indici secondari e offre la possibilità di aggiungere condizioni.
	\item Scan : offre la possibilità di fare query su attributi non chiave.
	\item UpdateItem : aggiorna un record esistente.
	\item DeleteItem : elimina uno specifico record dal database.
	\item BatchGetItem : reperisce record anche da più tabelle.
\end{itemize}  

\subsubsection*{MongoDb}
MongoDb è un DBMS molto utilizzato e memorizza i dati secondo il modello a documento. Anch'esso è schema less ed è stato pensato per applicazioni in continua evoluzione. Completamente gratuito deve essere ospitato da un provider di terze parti e di conseguenza le prestazioni dipendono dall'ambiente in cui è eseguito. MongoDb offre sia la possibilità di refernziare sia la possibilità di includere altre entità (documenti). Come altri DBMS nella categoria NoSql anche Mongo può essere eseguito su un ambiente distribuito per favorirne prestazioni e scalabilità. Ogni documento in una collezzione è identificato da un id univoco, tale id oltre che essere fondamentale per l'engine di mongo è molto utile per refernziare altri documenti o effettuare query complesse.
Diversamente da altri DBMS NoSql offre diverse tipologie di indici:
\begin{itemize}
	\item Indici secondari: utili a rendere performanati le ricerche
	\item Indici geospaziali: utili per calcolare distanze tra coordinate geospaziali
	\item Indici geohaystack: possono essere usatecome gli indici geospaziali ma permettono solo interrogazioni su superfici piane, dando prestazioni più elevate
	\item Indici testuali: permettono ricerche full-text
\end{itemize} 
MongoDb è una tecnologia molto usata e 

\begin{table}[htb]
\subsubsection*{DBMS a confronto}
\centering

\label{my-label}
\begin{tabular}{|l|l|l|}
\hline
Feature             & MongoDb                                                                                                                   & DynamoDb                                                                                                                    \\ \hline
Hosting             & \begin{tabular}[c]{@{}l@{}}Bare metal o \\ servizi di terze parti\end{tabular}                                            & Amazon                                                                                                                      \\ \hline
Data model          & Modello a documetni                                                                                                       & \begin{tabular}[c]{@{}l@{}}Modello chiave-valore\\ Modello a documenti\end{tabular}                                         \\ \hline
Licenza             & OpenSource                                                                                                                & Propietario                                  ma                                                                                \\ \hline
Chiavi/Indici       & \begin{tabular}[c]{@{}l@{}}Primary key,\\ Hashed index,\\ Geospatial index,\\ Geohystack index,\\ Text index\end{tabular} & \begin{tabular}[c]{@{}l@{}}Hash key,\\ Hash and range, \\ 5 global secondary index, \\ 5 local secondary index\end{tabular} \\ \hline
Modalità di accesso & Protocollo propietario                                                                                                    & RestAPI (HTTP)                                                                                                              \\ \hline
Trigger             & Si                                                                                                                        & Lambda AWS                                                                                                                  \\ \hline
Mapreduce           & Si                                                                                                                        & Si (EMR)                                                                                                                    \\ \hline
Gestione testo      & RegEx, Substring                                                                                                          & Substring                                                                                                                   \\ \hline
\end{tabular}


\end{table}

\subsection{Web services e server side framework}
Per favorire l'interoperabilità tra diverse applicazioni, eseguite in diverse piattaforme e/o in ambienti distribuiti come il web, il ``world wide web consortium" (W3C) ha delineato un architettura denominata ``Web service": 

%quoting
\begin{quotation}
\textit{A Web service is a software system designed to support interoperable machine-to-machine interaction over a network. It has an interface described in a machine-processable format (specifically WSDL). Other systems interact with the Web service in a manner prescribed by its description using SOAP messages, typically conveyed using HTTP with an XML serialization in conjunction with other Web-related standards.}
\end{quotation}
Tale architettura, definita come delle linee guida dal W3C, ha portato a quello che oggi è nota come architettura ``Representational State Transfer" (REST), introdotta per la prima volta nel 2000 da Roy Fielding. L'architettura REST si basa su un concetto fondamentale che è la risorsa (informazione) identificata da un URI. Ogni sistema software che vuole implementare l'architettura REST deve seguire le seguenti regole:
\begin{itemize}
	\item Client-server: Netta separazione dei ruoli tra client e server.
	\item Stateless: Nessuna informazione sullo stato del client deve essere salvata eliminando il vincolo di dover contattare sempre lo stesso server.
	\item Cacheable: Le risorse inviate dal server dovrebbero essere memorizzabili per un secondo utilizzo.
	\item Layered system: Sono previsti livelli intermedi come proxy per migliorare la scalabilità e aggiungere livelli di sicurezza.
	\item Uniform interface: Le interfacce di comunicazione dovrebbero appartenere tutte alla stessa famiglia (Es. HTTP).
\end{itemize}
Attualmente esistono molti framework che permettono di implementare l'architettura REST e i più noti sono:
\begin{itemize}
	\item NodeJs (Javascript)
	\item Laravel (PHP)
	\item Ruby on rails (Ruby)
\end{itemize}
In questo elaborato verrano prese in esame solo NodeJs e Laravel.
\subsubsection{NodeJS (Google event-driven javascript framework)}
Anche se javascript è nato come un linguaggio che doveva essere eseguito solo all'interno di un browser con lo sviluppo della versione 8 dell'interprete javascript di Chrome, noto browser di Google, le performance sono notevolmente migliorate al punto che sono nati framework server-side come NodeJs. NodeJs è un ``event-driven" framework che grazie alla natura asincrona di javascript riesce ad implementare in maniera soddisfacente il modello ``non-blocking I/O". Pensato per costruire applicazioni scalabili riesce a gestire numerose richeste simultaneamente e corredato di ``packet manager" gode della più grande libreria di moduli (relativi al framework) open-source del web.


\subsubsection{Laravel PHP}
Laravel è un framework PHP object oriented che si basa sull'architettura ``Model View Controller" permette di ottenere un ottima organizzazione sia logica che strutturale. I due concetti fondamentali di questo framework sono:
\begin{itemize}
	\item Routes: identifica una risorsa web.
	\item Controller: specifica un insieme di funzioni, utili ad organizzare il codice.

\end{itemize}
\section{Tecnologie selezionate}
Per lo sviluppo di questo progetto sono stati selezionati DynamoDb e NodeJs. DynamoDb poichè completamente ospitato e altamente scalabile e performante.
NodeJs invece per la sua natura non bloccante e basato sul modello ad eventi. Oltre alle tecnologie appena esamaminate sono state usate anche:
\begin{itemize}
	\item Elastic beanstalk: servizio AWS, basato su docker (container virtuali), permette scalabilità e ``continuos deploy". 
	\item Redis: all-in-memory key-value database, utilizzato come cache per i dati più richiesti.
	%\item AWS IoT: sistema di gestione per IoT basato sul protocollo MQTT.
\end{itemize}


\subsection{Elastic beanstalk}
Elasitic beanstalk è un servizio SaaS di AWS basato sulla tecnologia introdotta da ``Docker Inc." di virtualizzazione denominata ``container". Tale tecnologia ottimizza il concetto di virtualizzazione sfruttando lo stesso principio di isolamento delle risorse che viene usato per le macchine virtuali ma utilizza un approccio architetturale molto diverso: 

un container è un ambiente virtualizzato isolato per un'applicazione e include tutte le dipendeze necessarie all'esecuzione della stessa ma condivide il kernel con il resto dei container. Questa tecnologia, a differenza delle macchine virtuali, virtualizza solo l'applicazione e permette quindi di sfruttare al meglio la memoria offrendo una scalabilità orizzontale più efficente.

ELastic benastalk integra questa tecnologia con numerevoli feature di contorno: 
\begin{itemize}
	\item Deploy rapido tramite ``command line".
	\item Versioning dell'applicazione trasparente grazie al collegamento del container ad un repository gestito con un control version system (CVS).
	\item Scalabilità gestita in un paio di click: le istanze virtualizzate possono essere aumentate o diminuite secondo semplici regole.
	\item Permette lo swap tra container rendendo i deploy trasaprenti dall'esterno riducendo i downtime tra una versione e un'altra.
\end{itemize}

\subsection{Redis (All in-memory key-value DB)}
Redis è una tecnologia di datatbase NoSql denominato ``key/value store", risiede completamente in memoria Ram con persistenza su disco facoltativa. Le principali peculiarità di Redis sono:
\begin{itemize}
	\item Possibilità di esecuzione in ambiente distribuito
	\item Ad ogni chiave può essere assegnato un ``Time To Live" (TTL), utile se redis viene utilizzato come LRU cache.
	\item Meccanismo di publish/subscribe utile per la consistenza dei dati in ambienti distribuiti.
	\item Si possono eseguire transazioni: una sequenza di istruzioni che non varrà interrotta.
\end{itemize}
Le strutture supportate da redis sono:
\begin{enumerate}
	\item Stringhe
	\item Tabelle hash
	\item Liste
	\item Set
	\item Set ordinati
\end{enumerate}


\chapter{Progettazione software}


\section{Classificazione eventi}
Nell'ambito dell'attività di sviluppo del software di videoanalisi del progetto Parksmart sono stati implementati alcuni algoritmi di riconoscimento di particolari eventi, comprendenti lo stato di occupazione di singoli stalli di parcheggio. Il software cosi prodotto viene eseguito su sistemi embedded dedicati al fine di distribuire la computazione dell'attività di riconoscimento per informare l'utente su quale sia lo stallo di parcheggio libero più vicino. Allo stato attuale la classificazione degli eventi comprende esclusivamente la variazione di stato di uno stallo di parcheggio da libero a occupato e viceversa da occupato a libero, ma saranno successivamente introdotti alre categorie di eventi per ampliare il ventaglio di caratteristiche di prodotto sviluppate dal progetto Parksmart.
\vspace{5.7truecm}

\section{Database (DynamoDb AWS)}
Le entità che sono memorizzate nel DBMS in maniera persistente sono telecamere e stalli di parcheggio. Oltre a gli attributi definiti dai requisiti sono stati aggiunti alcuni attributi, membri di indici globali secondari, utili all'amministrazione. Data la tecnologia selezionata (DynamoDb) lo schema progettato ha la seguente struttura:

\vspace{0.5truecm}

\centerline{\includegraphics[width=\textwidth,height=\textheight,keepaspectratio]{./img/database_ps.png}}

\vspace{0.5truecm}

\centerline{\includegraphics[width=\textwidth,height=\textheight,keepaspectratio]{./img/database_cam.png}}

\vspace{0.5truecm}

Le operazioni che sono state implementate sono le seguenti:


\begin{itemize}
	\item Aggiornamento dello stato (Free/Busy) di uno specifico stallo di parcheggio.
	\item Richiesta di stalli di parcheggio di un geohash: da un geohash in input reperiscono tutti gli stalli di parcheggio che vi appartengono.
	\item Richiesta di stalli di parcheggio a partire da una posizione : data una posizione arbitraria si calcola il geohash e si reperiscono tutti gli stalli che vi appartengono.
	\item Richiesta di stalli di parcheggio a partire da un area geografica delimitata da una coppia di coordinate : date due coppie coordinate si calcolano tutti i geohash all'interno dell'area e si reperiscono tutti gli stalli che vi appartengono.
	\item Richiesta di videocamere a partire da un geohash : da un geohash in input reperiscono tutte le videocamere che vi appartengono.
	\item Richiesta di videocamere a partire da una posizione : data una posizione arbitraria viene calcolato il geohash e si reperiscono tutti le videocamere che vi appartengono.
	\item Richiesta di videocamere a partire da un area geografica delimitata da una coppia di coordinate : date due coppie coordinate si calcolano tutti i geohash all'interno dell'area e si reperiscono tutte le videocamere che vi appartengono.
	\item Richiesta di videocamere per modello : dato uno specifico modello di videocamera vengono reperite tutte le videocamere di quel modello.
	\item Richiesta di viedocamere per produttore : dato un produttore di videocamere si reperiscono tutte le viedeocamere vendute da quel produttore.
\end{itemize}

Date le operazioni appena descritte quelle che saranno utilizzate nell'architettura di interesse sono solo quelle relative a gli stalli di parcheggio.
\vspace{0.5truecm}


\section{NodeJs}
L'applicazione in nodejs è composta da due componenti fondamentali:
\begin{itemize}
	\item Moduli: librerie usate all'interno dell'applicazione.
	\item Routes: identificatori di risorse esposte all'esterno.
\end{itemize}
I moduli di terze parti che sono stai utilizzati comprendono ``Express js" un framework utile per il routing, ``ngeohash" libreria utile per codificare e decodificare dei geohash, ``AWS Sdk" che contiene tutte le librerie necessarie per comunicare con i servizi AWS (Es. DynamoDb), ``node-redis" modulo per eseguire le operazioni in cache e infine ``socket.io"+``socket.io-redis" per notificare gli eventi a gli utenti.
E' stato creato un modulo (``psm") ad-hoc per i nostri scopi che comprende i seguenti file:
\begin{enumerate}
	\item ParkingSpace.js con le seguenti funzioni ``pubbliche":
	\begin{itemize}
			\item updateParkingSpace(idPs) : aggiorna lo stato (Free/Busy) di uno specifico stallo di parcheggio.
			\item getPsByHash(hashSource): da un geohash in input reperiscono tutti gli stalli di parcheggio che vi appartengono.
			\item getPsByPosition(latitude, longitude) : data una posizione arbitraria si calcola il geohash e si reperiscono tutti gli stalli che vi appartengono.
			\item getPsByBounds(latHi, latLo, lngHi, lngLo) : date due coppie coordinate si calcolano tutti i geohash all'interno dell'area e si reperiscono tutti gli stalli che vi appartengono.
		\end{itemize}

	\item Camera.js con le seguenti funzioni ``pubbliche":
	\begin{itemize}
			\item getCamsByHash(hashSource) : da un geohash in input reperiscono tutte le videocamere che vi appartengono.
			\item getCamsByPosition(latitude, longitude) : data una posizione arbitraria viene calcolato il geohash e si reperiscono tutti le videocamere che vi appartengono.
			\item getCamsByBounds(latHi, latLo, lngHi, lngLo) : date due coppie coordinate si calcolano tutti i geohash all'interno dell'area e si reperiscono tutte le videocamere che vi appartengono.
			\item getCamsByModel (model): dato uno specifico modello di videocamera vengono reperite tutte le videocamere di quel modello.
			\item getCamsByVendor (vendor) : dato un venditore di videocamere si reperiscono tutte le viedeocamere vendute da quel produttore.
	\end{itemize}

	\item Websocket.js con le seguenti funzioni ``pubbliche":
	\begin{itemize}
		\item sendMessage(channel, message) : invia un messaggio in uno specifico namespace.
	\end{itemize}
\end{enumerate}

A seguire un frammento di codice dal file PerkingSpace.js

\begin{lstlisting}[caption=Code of getPsByPosition]

var AWS = require('./AWS_config.js').AWS_config();
var dynamo = require('dynamodb-doc');
var geohash = require('ngeohash');
var docClient = new dynamo.DynamoDB();
var async = require('async');



var asyncWorker = function(hash, returnData){

    this.returnData=returnData;
    this.hash=hash;
    this.getHash = function(callback){


        var params = {};
        params.TableName = 'Parking_space';
        params.IndexName = 'GeoHash';
        
        params.KeyConditions = [docClient.Condition('GeoHash', 'EQ', hash)];

        docClient.query(params, function(err, data){

            if(!err && data){
                Array.prototype.push.apply(returnData.Items, data.Items);
                
            }
            callback(err, true);
        });


    }
}


function ParkingSpace(){



	return {
		
		...
		...
		
		getPsByPosition: function(lat, lng, callback) {
		    
		    if(lat && lng && callback && typeof(callback)==='function'){
		        var hash = geohash.encode(lat, lng, 7)
		        
		        var hashArray = geohash.neighbors(hash, 7);
		        hashArray[hashArray.length]=hash;
		        var fnArray = [];
		        var returnData = {};
		        returnData.Items = [];
		        for (var i in hashArray){
		            fnArray[i]= (new asyncWorker(hashArray[i], returnData)).getHash;
		        }
		        async.parallel(fnArray, function(err){

	              	callback(null, returnData);
		           
		        });
		    }
		    else
		        callback('Invalid params', null);
		    
		    
		}, 

		...
	}
}

exports.ParkingSpace= ParkingSpace();


\end{lstlisting}

\vspace{0.5truecm}
Ogni utente, per accedere ad una risorsa esposta all'esterno, deve aggiungere nell'``header" della richiesta il token che gli è stato da cognito dato dopo aver effettuato il login. Quest'ultimo viene verificato con il sistema di autenticazione di AWS e infine salvato in cache. Le risorese esposte all'esterno per gli utenti sono le seguenti:

\lstset{language=HTTP}          

\begin{lstlisting}[frame=single]

HTTP 1.1 
GET /ps/getbyposition

Params:
{
	"lat" : Number,
	"lng" : Number
}

\end{lstlisting}

\vspace{1truecm}
\begin{lstlisting}[frame=single]

HTTP 1.1 
GET /ps/getbyhash

Params:
{
	"hashsource" : String
}

\end{lstlisting}


\begin{lstlisting}[frame=single]

HTTP 1.1 
GET /ps/getbybounds

Params:
{
	"lat_lo" : Number,
	"lng_lo" : Number,
	"lat_hi" : Number,
	"lng_hi" : Number
}

\end{lstlisting}

L'aggiornamento dello stato di occupazione di uno stallo di parcheggio viene effettuato sempre tramite restAPI, in fase di sviluppo le telecamere si stanno autenticando con una ``simple http auth" e di conseguenza ogni telecamera è fornita di username e password. Successivamente verrà usato il meccanismo dei certificati X.509: ogni telecamera sarà fornita di un certificato e sarà presente un database di ceritficati validi.
Per effettuare l'aggiornamento di uno stallo di parcheggio una telecamera deve effettuare una richiesta per la seguente risorsa:


\begin{lstlisting}[frame=single]

HTTPs 1.1 
GET /ps/update

Params:
{
	"IdPs" : String,
	"Free" : Boolean
}

\end{lstlisting}

Oltre all'identità della camera verrà verificata la relazione tra la videocamera e il parcheggio che deve eseere aggiornato di stato.






\section{Redis}
I vantaggi di usare una cache di supporto comune all'intero ``Auto scaling group" rende la condivisione di informazioni molto più semplice riuscendo cosi a salvare informazioni sullo stato dell'utente (es. ultimo token utilizzato) mantenendo comunque un modello stateless per l'applicazione. Per la strutturazione dei dati all'interno di redis viene sfruttata la funzione del ``path matching" cioè la possibilità di effettuare ricerche di chiavi con espressioni regolari. Uno stallo di parcheggio sarà memorizzato nel seguente modo:
\begin{quotation}
\textit{``geohash:idstallo"}
\end{quotation}
 e come valore sarà inserito lo stato dello stallo di parcheggio (0 free, 1 busy)

quindi per reperire tutti gli stalli di un geohash basta richiedere le chiavi 

\begin{quotation}
\textit{``geohash:*"}
\end{quotation}
 oppure  se si vuole ottenere lo stato del singolo stallo di parcheggio basta richiedere 
\begin{quotation}
\textit{``*:idstallo"}
\end{quotation} 
Le informazioni che vengono salvate sono le seguenti:

\begin{table}[htb]
\centering
\caption{Informazioni memorizzate su redis}
\label{my-label}
\begin{tabular}{|l|l|l|l|}
\hline
Chiave               & Struttura chiave & Valore               & Tipo valore \\ \hline
Stallo di parcheggio & geohash:idstallo & Stato di occupazione & Bit         \\ \hline
Token cognito        & idcognito:token  & TTL                  & Number      \\ \hline
\end{tabular}
\end{table}

Ad ogni valore chiave viene assegnato un TTL che viene aggiornato automaticamente ad ogni utilizzo della chiave, meccanismo utile per l'eliminazione dei record che utilizzati meno frequentemente.



\section{Lambda AWS}




%
\chapter{Architettura}
\noindent\begin{minipage}{1\textwidth}% adapt widths of minipages to your needs
\begin{center}


\end{center}
\end{minipage}


\section{Architettura 1 (NodeJs, Redis, Socket.io)}

\section{Architettura 2 (AWS API Gateway, AWS IoT, AWS Lambda)}

\section{Architetture a confronto}
\chapter{Workflow, Testing e tecnologie di supporto}
Ogni modulo sviluppato in questo progetto ha seguito un iter di validazione che comprende due fasi:
\begin{enumerate}
	\item Sviluppo
	\item Test
\end{enumerate}

Un aspetto importante della fase di sviluppo è stato l'utilizzo di un ``Version Control System" (VCS) un software di versioning di applicazioni molto utile nel lavoro in team. Il codice sorgente di ogni modulo sviluppato è mantenuto rispettivamente in un repository privato online gestito con ``Git", un vcs distribuito che gestisce tutte le versioni dei file. Ogni sviluppatore del team può scaricare il sorgente e lavorare con una copia in locale e pubblicare le modifiche appena completate. Grazie a questo tipo di software la gestione delle versioni del codice sorgente è semplificata, infatti esiste pure un meccanismo di tag che permette di identificare versioni dell'applicazione assegnando un tag a piacere.

La fase di test invece è stata caratterizzata da due fasi:
\begin{enumerate}
	\item Unit test
	\item HTTP test
\end{enumerate}
Per effettuare i test sono stati usati ``nodeunit" per lo unit test e ``supertest" per i test HTTP, entrambi moduli per nodejs che permettono di verificare che l'applicazione si comporti nel modo atteso. 

La fase di unit test si occupa di verificare che l'output delle funzioni sviluppate sia sempre corretto mentre invece la fase di test Http si occupa di verificare che l'applicazione risponda alle richieste HTTP in modo corretto.

\chapter{Conclusione}


Con lo sviluppo di questo progetto è stato creato un prototipo applicativo completo per la gestione real-time dell'occupazione degli stalli di parcheggio monitorati. Tale applicativo è tuttora in fase di evoluzione poichè si stanno valutando nuove tecnologie proposte da AWS in merito alla gestione della comunicazione tra i sistemi embedded e l'architettura cloud Park Smart. L'analisi dei requisiti è servita a tracciare le linee guida dello sviluppo e a tal proposito si è applicata una metodologia elastica di sviluppo del software, incentrata sul raggiungimento delle ``feature" di prodotto, con una ciclicità nella ridefinizione dell'architettura complessiva del sistema.

A tal proposito, i principi fondamentali della metodologia Agile sono stati di vitale importanza nella gestione complessiva del progetto.
In tal modo si è stati pronti a rispondere ai cambiamneti oltre che all'aderire alla pianificazione, modificando le priorità di lavoro nel rispetto dell'obiettivo finale \footnote{\href{http://www.agilemanifesto.org}{Agile Manifesto}}. 

Attualmente il software prodotto è impiegato attivamente all'interno del progetto Park Smart 	
\chapter{Sviluppi futuri}

Allo stato attuale il progetto Park Smart è ancora in fase di sviluppo e si stanno valutando nuove tecnologie per ampliare le ``feature" di prodotto e migliorare la \textit{user experience}. 

Attualmente si sta sviluppando il modulo per la comunicazione delle videocamere tramite MQTT autenticate con il Cloud ParkSmart con un certificato digitale X.509.
\newline

Verrà prodotta una dashboard web, con molteplici funzionalità collegate al cloud ParkSmart, dalla quale si potranno gestire tutte le informazioni su:
\begin{itemize}
	\item Telecamere;
	\item Stalli di parcheggio;
	\item Partner aziendali;
	\item Partner tecnici;
	\item Impianti installati;
\end{itemize}


Inoltre saranno previste funzionalità di monitoraggio per verificare lo stato degli impianti installati.


\end{document}





