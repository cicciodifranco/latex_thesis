\chapter{Obiettivi}
In questo elaborato verranno descritte le principali tecnologie usate e le valutazioni tecniche che hanno portato alle scelta delle stesse. Alcune scelte come il Service Provider o la specifica tecnologia di Database Management System (DBMS) sono state influenzate da fattori interni all'azienda, motivate non solo da fattori tecnologici ma anche da particolari partnership.


\section{Problematiche affrontate}
La creazione di un sistema real-time basato su tecnologie cloud per la gestione di eventi geolocalizzati necessita di diverse valutazioni ed in particolare sono stati sottoposti all'attenzione i seguenti punti:
\begin{itemize}
\item Scalabilità delle risorse
\item Latenza tra il verificarsi dell'evento e la notifica degli utenti
\item Sicurezza
\end{itemize}

\section*{Scalabilità delle risorse}
L'avanzare delle tecnologie in ambito informatico e delle telecomunicazioni ha apportato diversi cambiamenti nell'ambito dei sistemi distribuiti e, in particolare, alla nascita di quello che oggi chiamiamo Cloud.
Per cloud oggi si intende la modalità di strutturazione, orchestrazione ed erogazione {\itshape on-demand} di risorse informatiche tramite Internet come storage, capacità di elaborazione o hardware configurabile. Tali servizi sono completamente ospitati e configurati da provider che su richiesta e tramite procedure automatizzate, forniscono tali risorse in maniera rapida all'utente che le richiede.

I servizi cloud sono divisi in tre categorie fondamentali:
\begin{itemize}
\item SaaS (Software as a Service) - Servizi che consentono l'utilizzo di software installati su macchine remote
\item DaaS (Data as a Service) - Servizi di storage che consentono la memorizzazione persistente di dati su macchine remote
\item HaaS (Hardware as a Service) - Servizi che permettono l'elaborazione di dati in modo distribuito
\end{itemize}

I vantaggi che derivano dall'utilizzo di tali servizi sono la semplicità e la trasparenza con cui le risorse vengono erogate e/o aggiunte a sistemi già esistenti. Oggi la capacità di un sistema di adeguarsi al carico viene definita scalabilità, ovvero aumentare o diminuire le risorse in funzione delle necessità.

Tale proprietà viene considerata un parametro di qualità determinante in numerosi ecosistemi hardware-software. La distribuzione del carico e la massimizzazione delle prestazioni è un fattore chiave per sistemi software destinati all'utilizzo su vasta scala.

L'architettura proposta fa uso per lo più di servizi SaaS altamente scalabili e il provider scelto è Amazon le cui tecnologie, denominate Amazon Web Services (da questo momento AWS), sono tra le più utilizzate al mondo in ambito enterprise.


\section*{Latenza tra il verificarsi dell'evento e la notifica degli utenti}
Uno degli aspetti fondamentali per migliorare l'esperienza utente è il fattore real-time, cioè riuscire a visualizzare in tempo reale lo stato dello stallo di parcheggio interessato e nel contempo rimanere informato se lo stesso cambia di stato. 

La problematica più rilavante mentre si sta percorrendo un tragitto verso lo stallo interessato è appunto l'occupazione di quest'ultimo da parte di un'altro automobilista. 
La notifica del cambio di stato in real-time è un aspetto molto curato che permette alla logica interna dell'applicazione per smartphone di cercare lo stallo libero più vicino a quello interessato. 

Per contattare l'utente  in caso di un eventuale aggiornamento verrà usata la tecnologia ``web-socket'' che offre un canale di comunicazione {\itshape full-duplex} tra client e server rendendo cosi molto più semplici le comunicazioni dal server verso il client. 
In questo modo si riesce a diminuire l' I/O riducendo le richieste (chiamate alle API) da parte del client e consente al Backend in Cloud di contattare l'applicazione mobile dell'utente solo quando si verifichi un evento che di suo interesse.

\section*{Sicurezza}
Un aspetto che non può essere trascurato è quello della sicurezza e nello specifico le modalità di autenticazione e comunicazione.


Prima di poter iniziare una qualsiasi comunicazione ogni entità dovrà essere autenticata. Nello specifico le entità che dovranno comunicare sono due:

\begin{itemize}
	\item Utenti
	\item Telecamere
\end{itemize}

\subsection*{Autenticazione} 
Formalmente per autenticazione si intende quel processo tramite il quale una entità verifica che l'altra parte sia realmente chi dichiara di essere.
Le due modalità di autenticazione sono profondamente diverse a causa della diversa natura delle entità. 

Un utente per usufruire del servizio dovrà registrarsi e di conseguenza cedere alcune sue informazioni personali per essere autenticato. 
In ausilio a questa procedura verrà utilizzato un sevizio AWS denominato ``Cognito'' che permette, dopo aver effettuato il login su un qualsiasi identity provider (Facebook, Google, Park Smart stessa), di verificare l'identità dell'utente e mantenere la consistenza di un dataset personale utile alla sincronizzazione di più dispositivi. 
Cognito usa un meccanismo che verifica il Token fornito dal dispositivo utente con l'Identity Provider e, dopo un eventuale conferma, inizializza l'identità e la registra in un pool al quale fanno riferimento delle Access Control List (ACL) che specificano le capabilities della pool. In questo modo si ha la possibilità di gestire diversi gruppi di utenti assegnando o rimuovendo dinamicamente risorse specifiche.

L'autenticazione delle telecamere invece usa un meccanismo basato sulla certificazione digitale ``X.509'' che permette segretezza e autenticazione, tali certificati sono denominati ``self signed'' poiché rilasciati dall'azienda stessa ed hanno validità solo all'interno dell'ecosistema Park Smart. 
In tal modo ogni entità che vuole inviare un aggiornamento dovrà essere fornita di un certificato valido per completare l'Handshake iniziale. 
Grazie a questo meccanismo si garantisce segretezza nella comunicazione grazie alla crittografia asimmetrica data dallo standard X.509 e si riesce ad autenticare l'entità tramite la verifica del certificato.

\subsection*{Comunicazione}
Dopo che il processo di autenticazione è andato a buon fine l'utente potrà richiedere i dati relativi alla zona in prossimità della sua posizione. Il protocollo usato per questo tipo di comunicazioni sarà l' ``HTTPS'' (HTTP + TLS) che garantisce segretezza e integrità.

Per la comunicazione con le telecamere varrà usato un protocollo molto snello in termini di banda, denominato ``MQTT'', che usa un meccanismo di Publish/Subscribe e trova molte applicazioni nell ``Internet of Things" (IoT).
